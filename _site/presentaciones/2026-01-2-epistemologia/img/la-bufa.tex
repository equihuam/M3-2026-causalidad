% Options for packages loaded elsewhere
% Options for packages loaded elsewhere
\PassOptionsToPackage{unicode}{hyperref}
\PassOptionsToPackage{hyphens}{url}
\PassOptionsToPackage{dvipsnames,svgnames,x11names}{xcolor}
%
\documentclass[
  spanish,
  11pt,
  letterpaper,
  DIV=11,
  numbers=noendperiod]{scrartcl}
\usepackage{xcolor}
\usepackage{amsmath,amssymb}
\setcounter{secnumdepth}{-\maxdimen} % remove section numbering
\usepackage{iftex}
\ifPDFTeX
  \usepackage[T1]{fontenc}
  \usepackage[utf8]{inputenc}
  \usepackage{textcomp} % provide euro and other symbols
\else % if luatex or xetex
  \usepackage{unicode-math} % this also loads fontspec
  \defaultfontfeatures{Scale=MatchLowercase}
  \defaultfontfeatures[\rmfamily]{Ligatures=TeX,Scale=1}
\fi
\usepackage{lmodern}
\ifPDFTeX\else
  % xetex/luatex font selection
\fi
% Use upquote if available, for straight quotes in verbatim environments
\IfFileExists{upquote.sty}{\usepackage{upquote}}{}
\IfFileExists{microtype.sty}{% use microtype if available
  \usepackage[]{microtype}
  \UseMicrotypeSet[protrusion]{basicmath} % disable protrusion for tt fonts
}{}
\usepackage{setspace}
\makeatletter
\@ifundefined{KOMAClassName}{% if non-KOMA class
  \IfFileExists{parskip.sty}{%
    \usepackage{parskip}
  }{% else
    \setlength{\parindent}{0pt}
    \setlength{\parskip}{6pt plus 2pt minus 1pt}}
}{% if KOMA class
  \KOMAoptions{parskip=half}}
\makeatother
% Make \paragraph and \subparagraph free-standing
\makeatletter
\ifx\paragraph\undefined\else
  \let\oldparagraph\paragraph
  \renewcommand{\paragraph}{
    \@ifstar
      \xxxParagraphStar
      \xxxParagraphNoStar
  }
  \newcommand{\xxxParagraphStar}[1]{\oldparagraph*{#1}\mbox{}}
  \newcommand{\xxxParagraphNoStar}[1]{\oldparagraph{#1}\mbox{}}
\fi
\ifx\subparagraph\undefined\else
  \let\oldsubparagraph\subparagraph
  \renewcommand{\subparagraph}{
    \@ifstar
      \xxxSubParagraphStar
      \xxxSubParagraphNoStar
  }
  \newcommand{\xxxSubParagraphStar}[1]{\oldsubparagraph*{#1}\mbox{}}
  \newcommand{\xxxSubParagraphNoStar}[1]{\oldsubparagraph{#1}\mbox{}}
\fi
\makeatother


\usepackage{longtable,booktabs,array}
\usepackage{calc} % for calculating minipage widths
% Correct order of tables after \paragraph or \subparagraph
\usepackage{etoolbox}
\makeatletter
\patchcmd\longtable{\par}{\if@noskipsec\mbox{}\fi\par}{}{}
\makeatother
% Allow footnotes in longtable head/foot
\IfFileExists{footnotehyper.sty}{\usepackage{footnotehyper}}{\usepackage{footnote}}
\makesavenoteenv{longtable}
\usepackage{graphicx}
\makeatletter
\newsavebox\pandoc@box
\newcommand*\pandocbounded[1]{% scales image to fit in text height/width
  \sbox\pandoc@box{#1}%
  \Gscale@div\@tempa{\textheight}{\dimexpr\ht\pandoc@box+\dp\pandoc@box\relax}%
  \Gscale@div\@tempb{\linewidth}{\wd\pandoc@box}%
  \ifdim\@tempb\p@<\@tempa\p@\let\@tempa\@tempb\fi% select the smaller of both
  \ifdim\@tempa\p@<\p@\scalebox{\@tempa}{\usebox\pandoc@box}%
  \else\usebox{\pandoc@box}%
  \fi%
}
% Set default figure placement to htbp
\def\fps@figure{htbp}
\makeatother



\ifLuaTeX
\usepackage[bidi=basic]{babel}
\else
\usepackage[bidi=default]{babel}
\fi
% get rid of language-specific shorthands (see #6817):
\let\LanguageShortHands\languageshorthands
\def\languageshorthands#1{}


\setlength{\emergencystretch}{3em} % prevent overfull lines

\providecommand{\tightlist}{%
  \setlength{\itemsep}{0pt}\setlength{\parskip}{0pt}}



 


\KOMAoption{captions}{tableheading}
\makeatletter
\@ifpackageloaded{caption}{}{\usepackage{caption}}
\AtBeginDocument{%
\ifdefined\contentsname
  \renewcommand*\contentsname{Tabla de contenidos}
\else
  \newcommand\contentsname{Tabla de contenidos}
\fi
\ifdefined\listfigurename
  \renewcommand*\listfigurename{Listado de Figuras}
\else
  \newcommand\listfigurename{Listado de Figuras}
\fi
\ifdefined\listtablename
  \renewcommand*\listtablename{Listado de Tablas}
\else
  \newcommand\listtablename{Listado de Tablas}
\fi
\ifdefined\figurename
  \renewcommand*\figurename{Figura}
\else
  \newcommand\figurename{Figura}
\fi
\ifdefined\tablename
  \renewcommand*\tablename{Tabla}
\else
  \newcommand\tablename{Tabla}
\fi
}
\@ifpackageloaded{float}{}{\usepackage{float}}
\floatstyle{ruled}
\@ifundefined{c@chapter}{\newfloat{codelisting}{h}{lop}}{\newfloat{codelisting}{h}{lop}[chapter]}
\floatname{codelisting}{Listado}
\newcommand*\listoflistings{\listof{codelisting}{Listado de Listados}}
\makeatother
\makeatletter
\makeatother
\makeatletter
\@ifpackageloaded{caption}{}{\usepackage{caption}}
\@ifpackageloaded{subcaption}{}{\usepackage{subcaption}}
\makeatother
\makeatletter
\@ifpackageloaded{tcolorbox}{}{\usepackage[skins,breakable]{tcolorbox}}
\makeatother
\makeatletter
\@ifundefined{shadecolor}{\definecolor{shadecolor}{named}{black}}{}
\makeatother
\makeatletter
\makeatother
\makeatletter
\ifdefined\Shaded\renewenvironment{Shaded}{\begin{tcolorbox}[breakable, interior hidden, sharp corners, borderline west={3pt}{0pt}{shadecolor}, enhanced, frame hidden, boxrule=0pt]}{\end{tcolorbox}}\fi
\makeatother
\usepackage{bookmark}
\IfFileExists{xurl.sty}{\usepackage{xurl}}{} % add URL line breaks if available
\urlstyle{same}
\hypersetup{
  pdftitle={Nota ejecutiva de solicitud de información estratégica},
  pdfauthor={Miguel Eq.},
  pdflang={es},
  colorlinks=true,
  linkcolor={blue},
  filecolor={Maroon},
  citecolor={Blue},
  urlcolor={Blue},
  pdfcreator={LaTeX via pandoc}}


\title{Nota ejecutiva de solicitud de información estratégica}
\usepackage{etoolbox}
\makeatletter
\providecommand{\subtitle}[1]{% add subtitle to \maketitle
  \apptocmd{\@title}{\par {\large #1 \par}}{}{}
}
\makeatother
\subtitle{Insumos prioritarios para el análisis de viabilidad de la
declaratoria de los Cerros}
\author{Miguel Eq.}
\date{2026-01-12}
\begin{document}
\maketitle


\setstretch{1.2}
\subsection{Propósito}\label{propuxf3sito}

Identificar información clave que permita \textbf{comprender el alcance
institucional}, recuperar \textbf{antecedentes técnicos y sociales}, y
fortalecer la \textbf{viabilidad territorial y social} del proceso de
análisis para una eventual declaratoria de protección de los Cerros.

\begin{center}\rule{0.5\linewidth}{0.5pt}\end{center}

\subsection{1. Alcance político e
institucional}\label{alcance-poluxedtico-e-institucional}

\textbf{Se solicita contar con:}

\begin{itemize}
\tightlist
\item
  Lista de \textbf{categorías de Áreas Naturales Protegidas (ANP)}
  consideradas preliminarmente.
\item
  Definición del \textbf{alcance político e institucional} del proceso
  de declaratoria.
\item
  Relación de \textbf{responsabilidades específicas de SAMA}.
\item
  \textbf{Listado o base de datos de dependencias estatales y
  municipales} involucradas o a involucrar, con sus competencias.
\end{itemize}

\begin{center}\rule{0.5\linewidth}{0.5pt}\end{center}

\subsection{2. Antecedentes técnicos, sociales y políticos del
territorio}\label{antecedentes-tuxe9cnicos-sociales-y-poluxedticos-del-territorio}

\textbf{Con el fin de evitar duplicaciones de esfuerzo y reactivación de
conflictos:}

\begin{itemize}
\tightlist
\item
  Informes y estudios de \textbf{intentos previos de declaratoria} u
  otras figuras de protección (estudios, resultados, resoluciones).
\item
  \textbf{Registro o base de datos de conflictos sociales} documentados
  o históricos asociados al territorio.
\item
  Documentación institucional sobre \textbf{lecciones aprendidas} de
  procesos anteriores.
\end{itemize}

\begin{center}\rule{0.5\linewidth}{0.5pt}\end{center}

\subsection{3. Dinámicas socioantropológicas y viabilidad
territorial}\label{dinuxe1micas-socioantropoluxf3gicas-y-viabilidad-territorial}

\textbf{En caso de existir:}

\begin{itemize}
\tightlist
\item
  Estudios o diagnósticos \textbf{socioantropológicos} del área.
\item
  Información sobre \textbf{actores sociales locales}, usos del
  territorio y procesos organizativos relevantes.
\end{itemize}

\begin{center}\rule{0.5\linewidth}{0.5pt}\end{center}

\subsection{4. Permisos y autorizaciones
urbanas}\label{permisos-y-autorizaciones-urbanas}

\textbf{Se propone solicitar:}

\begin{itemize}
\tightlist
\item
  Relación de \textbf{permisos de construcción autorizados} en el área
  de los Cerros, a nivel municipal y estatal.
\item
  Información asociada disponible: ubicación, tipo de obra, fecha y
  dependencia responsable.
\end{itemize}

\begin{center}\rule{0.5\linewidth}{0.5pt}\end{center}

\subsection{5. Instrumentos de planeación y
cartografía}\label{instrumentos-de-planeaciuxf3n-y-cartografuxeda}

\textbf{Documentos estratégicos:}

\begin{itemize}
\tightlist
\item
  \textbf{Programa Estatal de Desarrollo Urbano y Ordenamiento Ecológico
  del Estado}, incluyendo anexos técnicos y cartografía.
\item
  Programas municipales, ordenamientos ecológicos locales y
  zonificaciones vigentes (de existir).
\end{itemize}

\textbf{Información espacial:}

\begin{itemize}
\tightlist
\item
  Cartografía oficial del área (uso de suelo, tenencia, zonificación).
\item
  Capas de ANP existentes o propuestas, expansión urbana y proyectos
  autorizados.
\end{itemize}

\begin{center}\rule{0.5\linewidth}{0.5pt}\end{center}

\subsection{6. Información a solicitar con
cautela}\label{informaciuxf3n-a-solicitar-con-cautela}

\emph{(Tema sensible, sugerido para abordarse en el momento menos
confrontativo)}

\begin{itemize}
\tightlist
\item
  Información, estudios o análisis sobre \textbf{plusvalía actual y
  proyectada} de los terrenos en los Cerros.
\item
  Datos sobre \textbf{presión inmobiliaria, especulación del suelo} y
  cambios recientes en el valor de la tierra asociados al crecimiento
  urbano.
\item
  Identificación preliminar de \textbf{actores cuyo interés principal
  esté vinculado al valor del suelo}, más que al uso actual del
  territorio.
\end{itemize}

\begin{center}\rule{0.5\linewidth}{0.5pt}\end{center}

\subsection{Cierre}\label{cierre}

La información solicitada permitirá \textbf{fortalecer la coordinación
interinstitucional}, anticipar riesgos sociales y territoriales, y
apoyar una \textbf{toma de decisiones informada y viable} para el
proceso.

\begin{center}\rule{0.5\linewidth}{0.5pt}\end{center}

Si quieres, puedo:

\begin{itemize}
\tightlist
\item
  Ajustar aún más la \textbf{extensión para que quepa estrictamente en
  una página PDF} (márgenes, interlineado).
\item
  Preparar una \textbf{versión tipo checklist} para llevar impresa a la
  reunión.
\item
  Redactar un \textbf{oficio formal} derivado de esta nota.
\end{itemize}




\end{document}
